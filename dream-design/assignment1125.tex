\documentclass{article}

% The geometry package allows for easy page formatting.
\usepackage{geometry}
\geometry{letterpaper}

% Load up special logo commands.
\usepackage{doc}

% Package for formatting URLs.
\usepackage{url}

% Packages and definitions for graphics files.
\usepackage{graphicx}
\usepackage{epstopdf}
\DeclareGraphicsRule{.tif}{png}{.png}{`convert #1 `dirname #1`/`basename #1 .tif`.png}
%
% Set the title, author, and date.
%
\title{A ``Dream'' Interface for the Ultimate Smartwatch}
\author{Joaquin Loustau}
\date{November 27, 2014}

%
% The document proper.
%
\begin{document}

% Add the title section.
\maketitle

% Add an abstract.
\abstract{
}

% Add various lists on new pages.
\pagebreak
\tableofcontents

% Start the paper on a new page.
\pagebreak

%
% Body text.
%
\section{Description}
The wristwatch has been an ever-present wearable technology since its inception in the early 1900s, and has undergone continuous technical refinement since that time. Researchers have long viewed the immediacy and ubiquity of the wristwatch as a vehicle for computation, pushing its capabilities to ever-greater heights. In 2000, IBM demonstrated the first watch running a full operating system.  

One of the main reasons a wristwatch emerges as an attractive wearable device in the fact that a large fraction of the population is already accustomed to wearing wristwatches, thus favoring the learning curve for novice users. Furthermore, people generally keep watches on their wrists, and watches are less prone to be misplaced compared to phones and tablets. For example, a hip holster or pocket are not convenient places to keep a cellular phone while sitting in a car, and so people tend to keep them in the car seat and forget them when they leave the car in the parking lot. Another significant advantage of a wrist watch is that it is much more accessible than many of the other devices one may carry. It is often said that one of the reasons for the initial success of the Palm was its moving to an instant-on paradigm, i.e. eliminating the long boot up time associated with laptops. Wrist watches move users to the next step: an instantly-viewable paradigm.

From its beginnings, wearable devices have often faced a major challenge: user interface. An adequate smart phone user interface was missing for a long time. An adequate UI was the major success factor of the iPhone --the reason it was superior to all other approaches at that time. The same challenge is ahead for this new class of device: the smartwatch. 

Unlike smartphones, which can be scaled to a variety of sizes, smartwatches must be small and unobtrusive in order to remain socially acceptable, which has long limited their practicality. The reduced size of the screen poses challenges to the possibilities and usability of multi-touch technology. One approach, followed by Google, is voice recognition.  However, it does not appear to be socially acceptable to talk to a watch yet. In addition, in environments with strong noise pollution, its accuracy falls dramatically. This is particularly true of devices such as the Samsung Gear, as its microphones are tuned to support surrounding noise for its video recording feature.

Devices such as the Moto 360 and Apple Watch make use of a gyroscope to recognize when users turn their wrist to look at the device so as to lighten up the watch face. This feature provides the foundations for a technology which emulates the most intuitive movements we do with our hands and arms - touching and pointing to something. 

The User Interface design below aims to enhance the most important capabilities of existing smartwatches while eliminating aspects and features that proved to be an impediment for usability purposes.

\section{Design}
The watch designed, named The Watch, stays true to the timeless form of the classic wristwatch. A round design maximizes the display area and, unlike the Moto 360, effectively makes use of 100% of the screen.

In order to expand accessibility, two almost identical versions of The Watch accommodate both users that use their watch on their left arm (usually right-handed) and those who wear it on their right hand (frequently left-handed). The only difference between the two models is that the digital crown and fingerprint reader tip are exchanged -the digital crown is on the same side of the watch as the hand that is free.

Since wristwatches were invented in the 19th century, people have been glancing at them to check the time. With The Watch, this simple, reflexive act allows users to learn so much more. The Watch offers users the possibility to access scannable summaries of the information they seek out most frequently. These will very similar to the cards in the Context Stream: short snippets of relevant information, with an optional photo backdrop. Users will be able to swipe up to see expanded information. 

 To switch between cards, users simply swipe to the sides from the current screen. Users can tap on one of these cards to go directly to its corresponding app for more details. This makes navigation fluid and responsive.

The Watch has a built-in ambient light sensor -alike the one found in the Moto 360. When it is enabled, the device adapts the light of the screen according to the environment the user is in, saving the user having to access the settings to increase or decrease the screen brightness whenever the lighting conditions change. 
The Watch organizes users’ information by predicting what they need, when they need to see it and displaying it before they even ask. Users receive real-time notifications -in the form of a gentle tap- for incoming mail, messages, and calls. From the watch, they are able to decide whether to dismiss them or answer them (either on a phone or on the watch itself).  If the user decides to answer a text message on the phone, he will have three options. Firstly, he will be able to choose from a list of 4 or 5 text messages produced by the watch’s software based on the incoming message. For example, if the incoming message includes the words “What time...,” the sample messages will most likely be an array of times of the day. Secondly, if none of the custom messages offered by The Watch satisfies the user, he is able to select the option to answer by voice command, by tapping on the microphone that will be displayed on the screen. Lastly, the user is able to choose from a diverse group of emoji available. 
In terms of user interaction, The Watch offers five (5) different ways for the user to interact with the device: 
\begin{enumerate}
\item \textbf{Motion Recognition}
A 3-D gesture-recognition chip will be included inside The Watch that will enable it to recognize complex and precise hand motions and produce content accordingly. The chip uses sonar via an array of ultrasound transducers --small acoustic resonators-- that send ultrasonic pulses outward in a hemisphere, echoing off any objects in their path. Those echoes come back to the transducers, and the elapsed time is measured by a connected electronic chip. When using a two-dimensional array of transducers, the time measurements can be used to detect a range of hand gestures in three dimensions within a distance of about a meter. 

Specific examples of motion recognition include making and declining phone calls. When the user receives a call, he is able to decline the call by just shaking The Watch. The Watch will also offer Hands on Talk, the technology for bouncing sound waves off the user’s palm when making a call with the watch. 

\item \textbf{Touch Screen}
In addition to recognizing touch, The Watch senses force, adding a new dimension to the user interface. The Force Touch technology, currently used by the Apple Watch, uses extremely small electrodes around the display to distinguish between a light tap and a deep press. This interaction triggers instant access to a range of contextually specific controls -such as an action menu in Messages, or a mode that allows the user to select different watch faces -- whenever the user wants. 
Users can swipe downwards on the screen to cancel a selection or back one screen -when one screen away from the watch face pressing the digital crown and swiping down will have the same effect- and sideways to switch between applications.  

\item \textbf{Voice Recognition }
The Watch's Voice recognition will provide an accurate and reliable recognition that will allow the user to perform every single action the phone permits -except for authorizing payments- through voice commands. This attempts to provide full functionality to visually impaired users. 

Since The Watch will not include a video recording feature, the microphone will be configured so as to isolate as much environment noise as possible.

\item \textbf{Digital Crown}
This unique input device introduced by the Apple Watch offers phenomenal ``out of the box'' possibilities. On mechanical watches, the crown has historically been used to set the time and date and to wind the mainspring. The Digital Crown allows users to zoom and scroll nimbly and precisely, without obstructing their view. Users can also push it like a button to return to the watch face, making it an integral part of the The Watch experience. 

\item \textbf{Fingerprint Scanner}
One of the sides of The Watch has a built-in fingerprint reader tip, similar to those often found on Lenovo Laptops. This feature serves a single but crucial purpose: it allows users to verify their identity when making payments using the watch. 

A significant flaw of present-day smart watches is that when they run out of battery, users do not simply lose just the smart function. The display shuts down, and users lose the ability to even tell time. At this point, devices such as the Moto 360 become nothing more than an expensive bracelet. The Watch presents an innovative solution to this problem. Once the battery dies, the watch behaves like an automatic digital watch -the natural movement of the user's arm provides enough energy to keep the watch face and display time accurately. Seiko, Citizen and more recently Ventura have manufactured several generations of automatic digital watches, and extending this technology to a smart watch seems feasible.

Finally, in terms of battery charging, The Watch will use a wireless charger employing inductive charging. This mechanism for charging was selected in order to maintain The Watch's waterproofness. 
\end{enumerate}

\section{Usage Scenarios}
\section{Rationale}
\section{Usability Metrics Forecast}

\bibliography{assignment1016}
\bibliographystyle{plain}

\end{document}
