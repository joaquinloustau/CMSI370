\documentclass{article}

% The geometry package allows for easy page formatting.
\usepackage{geometry}
\geometry{letterpaper}

% Load up special logo commands.
\usepackage{doc}

% Package for formatting URLs.
\usepackage{url}

% Packages and definitions for graphics files.
\usepackage{graphicx}
\usepackage{epstopdf}
\DeclareGraphicsRule{.tif}{png}{.png}{`convert #1 `dirname #1`/`basename #1 .tif`.png}
%
% Set the title, author, and date.
%
\title{A ``Dream'' Interface for the Ultimate Smartwatch}
\author{Joaquin Loustau}
\date{November 27, 2014}

%
% The document proper.
%
\begin{document}

% Add the title section.
\maketitle

% Add an abstract.
\abstract{
}

% Add various lists on new pages.
\pagebreak
\tableofcontents

% Start the paper on a new page.
\pagebreak

%
% Body text.
%
\section{Description}
The wristwatch has been an ever-present wearable technology since its inception in the early 1900s, and has undergone continuous technical refinement since that time. Researchers have long viewed the immediacy and ubiquity of the wristwatch as a vehicle for computation, pushing its capabilities to ever-greater heights. In 2000, IBM demonstrated the first watch running a full operating system.  

One of the main reasons a wristwatch emerges as an attractive wearable device in the fact that a large fraction of the population is already accustomed to wearing wristwatches, thus favoring the learning curve for novice users. Furthermore, people generally keep watches on their wrists, and watches are less prone to be misplaced compared to phones and tablets. For example, a hip holster or pocket are not convenient places to keep a cellular phone while sitting in a car, and so people tend to keep them in the car seat and forget them when they leave the car in the parking lot. Another significant advantage of a wrist watch is that it is much more accessible than many of the other devices one may carry. It is often said that one of the reasons for the initial success of the Palm was its moving to an instant-on paradigm, i.e. eliminating the long boot up time associated with laptops. Wrist watches move users to the next step: an instantly-viewable paradigm.

From its beginnings, wearable devices have often faced a major challenge: user interface. An adequate smart phone user interface was missing for a long time. An adequate UI was the major success factor of the iPhone --the reason it was superior to all other approaches at that time. The same challenge is ahead for this new class of device: the smartwatch. 

Unlike smartphones, which can be scaled to a variety of sizes, smartwatches must be small and unobtrusive in order to remain socially acceptable, which has long limited their practicality. The reduced size of the screen poses challenges to the possibilities and usability of multi-touch technology. One approach, followed by Google, is voice recognition.  However, it does not appear to be socially acceptable to talk to a watch yet. In addition, in environments with strong noise pollution, its accuracy falls dramatically. This is particularly true of devices such as the Samsung Gear, as its microphones are tuned to support surrounding noise for its video recording feature.

Devices such as the Moto 360 and Apple Watch make use of a gyroscope to recognize when users turn their wrist to look at the device so as to lighten up the watch face. This feature provides the foundations for a technology which emulates the most intuitive movements we do with our hands and arms - touching and pointing to something. 

The User Interface design below aims to enhance the most important capabilities of existing smartwatches while eliminating aspects and features that proved to be an impediment for usability purposes.

\section{Design}
\section{Usage Scenarios}
\section{Rationale}
\section{Usability Metrics Forecast}

\bibliography{assignment1016}
\bibliographystyle{plain}

\end{document}
